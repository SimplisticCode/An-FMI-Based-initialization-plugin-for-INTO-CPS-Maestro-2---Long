% This is samplepaper.tex, a sample chapter demonstrating the
% LLNCS macro package for Springer Computer Science proceedings;
% Version 2.20 of 2017/10/04
%
\documentclass[runningheads]{llncs}
\usepackage{graphicx}
\usepackage{ae,aecompl}
\usepackage[utf8]{inputenc}
\usepackage[english]{babel}
\usepackage{verbatim}
\usepackage{graphicx}
\usepackage{amsfonts}
\usepackage{amsmath}
\usepackage{amssymb}
\usepackage{stmaryrd}
\usepackage{amstext}
\usepackage{bm} 
\let\proof\relax
\let\endproof\relax
\usepackage{amsthm}
\usepackage{siunitx}
\usepackage{mathrsfs}
\usepackage{wrapfig}
%\usepackage{minted}
\usepackage{algorithm}
\usepackage{algpseudocode}
\usepackage{semantic}
\usepackage[dvipsnames]{xcolor}
\usepackage{paralist}
\usepackage{cite}
\usepackage[capitalise,nameinlink]{cleveref}
\usepackage{url}
\usepackage{pgfplots}
\pgfplotsset{compat=1.17} 

\usepackage{tikz}
\usepackage{environ}

\usepackage{cleveref}
\usepackage{listings}
\newcommand{\orcid}[1]{\href{https://orcid.org/#1}{\includegraphics[width=10pt]{images/orcid.png}}}
% Used for displaying a sample figure. If possible, figure files should
% be included in EPS format.
%
% If you use the hyperref package, please uncomment the following line
% to display URLs in blue roman font according to Springer's eBook style:
% \renewcommand\UrlFont{\color{blue}\rmfamily}
\input{commands}
\input{case_study_results}

\definecolor{codegreen}{rgb}{0,0.6,0}
\definecolor{codegray}{rgb}{0.5,0.5,0.5}
\definecolor{codepurple}{rgb}{0.58,0,0.82}
\definecolor{backcolour}{rgb}{0.95,0.95,0.92}

\lstdefinestyle{mystyle}{
    backgroundcolor=\color{backcolour},   
    commentstyle=\color{codegreen},
    keywordstyle=\color{magenta},
    numberstyle=\tiny\color{codegray},
    stringstyle=\color{codepurple},
    basicstyle=\ttfamily\footnotesize,
    breakatwhitespace=false,         
    breaklines=true,                 
    captionpos=b,                    
    keepspaces=true,                 
    numbers=left,                    
    numbersep=5pt,                  
    showspaces=false,                
    showstringspaces=false,
    showtabs=false,                  
    tabsize=2
}
\lstset{style=mystyle}


\begin{document}

\NewEnviron{scaletikzpicturetowidth}[1]{%
  \def\tikz@width{#1}%
  \def\tikzscale{1}\begin{lrbox}{\measure@tikzpicture}%
  \BODY
  \end{lrbox}%
  \pgfmathparse{#1/\wd\measure@tikzpicture}%
  \edef\tikzscale{\pgfmathresult}%
  \BODY
}
%
\title{An FMI-Based Initialization Plugin for INTO-CPS Maestro 2\thanks{We are grateful to the Poul Due Jensen Foundation, which has supported the establishment of a new Centre for Digital Twin Technology at Aarhus University. Finally, we thank the reviewers for the thorough feedback.}}

%
%\titlerunning{Abbreviated paper title}
% If the paper title is too long for the running head, you can set
% an abbreviated paper title here
%
\author{Simon Thrane Hansen\inst{1} \orcidID{0000-0002-3796-4319} \and
Casper Thule\inst{1} \orcidID{0000-0001-6606-9236} \and
Cláudio Gomes \inst{1} \orcidID{0000-0003-2692-9742}}
%
\authorrunning{S. Thrane et al.}
% First names are abbreviated in the running head.
% If there are more than two authors, 'et al.' is used.
%
\institute{DIGIT, Department of Engineering, Aarhus University, \email{\{sth, casper.thule, claudio.gomes\}@eng.au.dk\}}}
%

\maketitle              % typeset the header of the contribution
%

\begin{abstract}
The accuracy of the result in a co-simulation is dependent on the correct initialization of all the simulation units. In this work, we consider co-simulation where the simulation units are described as Functional Mock-up Units (FMU).
The Functional Mock-up Interface specification specifies constraints to the initialization of variables in the scope of a single FMU. However, it does not consider the initialization of interconnected variables between instances of FMUs. Such interconnected variables place particular constraints on the initialization order of the FMUs.\\
The approach taken to calculate a correct initialization order is based on predicates from the FMI specification and the topological ordering of both internal connections and interconnected variables. The approach supports the initialization of algebraic loops using fixed point iteration. %  dependencies between FMU variables are dismissed. %This approach has been compared to other already existing approaches for FMI initialization. 
The approach has been realized as a plugin for the open-source INTO-CPS Maestro 2 Co-simulation framework. It has been tested for various scenarios and compared to an existing \textit{Initializer} that has been validated through academic and industrial application.% The approach has also been directly tested against an established co-simulation algorithm generator implemented in Prolog.

\keywords{Co-simulation \and Initialization \and Algebraic loop \and Topological ordering \and FMI}
\end{abstract}

%The correct initialization of a co-simulation depends not only on the value being assigned to each port but also the order in which the ports are initialized.





\section{Introduction}\label{sc:introduction}
Cyber-physical systems (CPS) are becoming ever more sophisticated, while market pressure shortens the available development time. One of the tools to manage the increasing complexity of such systems is co-simulation since it tackles their heterogeneous nature. Co-simulation is a technique to combine multiple black-box simulation units to compute the behavior of the combined models as a discrete trace (see, e.g., \cite{Kubler2000, Gomes2018}). The simulation units, often developed independently from each other, are coupled using a master algorithm, also often developed independently, that communicates with each simulation unit via its interface. This interface comprises functions for setting/getting inputs/outputs, and computing the associated model behavior over a given time interval.
The Functional Mock-up Interface (FMI) standard \cite{Blochwitz2012, fmi_2019} is such an interface prescribing how to communicate with each simulation unit. The interface is used to connect different simulation units, called Functional Mock-up Units (FMUs), exchange values between them, and make them progress in time.

A typical co-simulation consists of three phases: initialization, simulation, and teardown \cite{Thule2019b}. This work concentrates on the first. The FMI specifies criteria for how a single FMU shall be initialized. However, the FMI is not concerned with how a connected system of multiple FMUs is initialized correctly as a whole.

The way a system of multiple FMUs should be initialized and interacted with depends on each FMUs implementation and interconnections to other FMUs \cite{gomes_lucio_vangheluwe_2019}, since these place precedence constraints between the FMU variables. These constraints can introduce algebraic loops, which places particular requirements on the initialization order to calculate the initial values of the variables in the algebraic loop\cite{Bastian2011a}. Algebraic loops occur whenever an interconnected FMU variable indirectly depends on itself. Not solving an algebraic loop can lead to a prohibitively high error in the co-simulation result \cite{Arnold2014}, and invalid results, as shown in \cref{sec:case_study}. For variables that do not belong in an algebraic loop, the initialization has to ensure that a variable is never read before it is set like the classical \textit{readers–writers problem}. For variable within an algebraic loop, the initialization has to make sure that all initial values have converged to a fixed point.

Other approaches for the generation of co-simulation algorithms have avoided co-simulation scenarios containing algebraic loops since their presence reduces the chance of obtaining a deterministic co-simulation result\cite{Amalio2016CheckingCo-simulation, BromanCompositionCo-Simulation, Gomes2019c}. This choice is driven by the fact that not all co-simulation scenarios containing algebraic loops are valid since those algebraic loops never converge, or might converge to unexpected solutions. However, as shown in \cref{sec:case_study}, algebraic loops can be essential to obtaining valid simulation results, and a well-established co-simulation framework should be able to handle these scenarios. 

\textit{Contribution:} This paper describes an approach for calculating the initialization order of an FMI-based co-simulation in linear time of the number of interconnected variables, even when algebraic loops are present.
% The approach complies with the semantics of FMI and support the initialization of algebraic loops between interconnected FMU variables and identifies divergence in a co-simulation scenario. 
The approach does not put any constraints on choosing a master algorithm that should be used to carry out the simulation. 
This means that the approach is suitable to combine with well-established master algorithms like Gauss-Seidel and Jacobian \cite{Palensky2017}. 
\claudio{Maybe move the above sentence to the conclusion as it's kind of redundant with the previous sentence, and seems to a conclusion of this approach.}
The approach is realized as a plugin to the co-simulation framework called INTO-CPS Maestro 2 (Maestro 2), introduced in \cite{Thule2019b}.
The realized plugin has been tested for various co-simulation scenarios and compared to an existing approach that has been validated through academic and industrial applications. 
Furthermore, the calculated initialization order is systematically verified by the semantics of co-simulation introduced in \cite{gomes_lucio_vangheluwe_2019,Gomes2019c}. \\
\claudio{Simon, can you please remove the double slashes (like the one right before this comment) from the end of sentences and paragraphs? It's bad practice to force Latex to make new lines. Also, you don't want to worry about that... latex usually does a good job at laying out the text for you.}

\textit{Structure:} The paper is structured as follows: Section \ref{sc:background} gives a brief background of the formalization of FMUs and Maestro 2. Section \ref{sc:initilization} describes the approach taken to calculate the initialization order. It is followed by section \ref{sc:implementation}, where the realization of the approach is presented. Finally, section \ref{sc:summary} provides concluding remarks and describes future work.
\claudio{Simon, please make sure that the references are according to what the template suggests. Some templates ask references like \emph{Section~\ref{sc:summary}} and others ask for \emph{Sect.~\ref{sc:summary}}, or \emph{section~\ref{sc:summary}}, or \emph{sect.~\ref{sc:summary}}. The package cleveref allows you to configure how you want references to be displayed (see the package documentation) and make this uniform throughout all document. The same comment applies to figures, tables, definitions, etc. If in doubt on how to make the references, copy the style in the template text.}

\section{Background}\label{sc:background}
In this section, we provide a formalization of FMI co-simulation and a brief background on INTO-CPS Maestro 2.

\subsection{FMU definitions}
To describe the formalization of FMUs, we adopt the vocabulary from \cite{gomes_lucio_vangheluwe_2019, Gomes2018}. The main definitions of relevance to this paper will be presented, but readers are referred to the original publications for more information. This paper is only concerned with the initialization-phase of a co-simulation, making time of an FMU irrelevant. The formalization from Gomes et al. is extended with new definitions regarding algebraic loops, and convergence of fixed point iteration.
\begin{definition}[FMU]\label{def:fmu}
  An FMU with identifier $c$ is represented by the tuple   
  $$\tuple{\stateset{c}, \inputs{c}, \outputs{c}, \fset{c}, \fget{c}},$$
  where:
  \begin{inparadesc}
    \item $\stateset{c}$ represents the state space;
    \item $\inputs{c}$ and $\outputs{c}$ the set of input and output variables, respectively;
    \item $\fset{c} : \stateset{c} \times \inputs{c} \times \values \to \stateset{c}$ and $\fget{c}: \stateset{c} \times \outputs{c} \to \values$ are functions to set the inputs and get the outputs, respectively (we abstract the set of values that each input/output variable can take as $\values$).
  \end{inparadesc}
\end{definition}

\begin{definition}[Scenario]\label{def:cosim_scenario}
  A scenario is a structure $\tuple{\fmus, \coupling}$ where each identifier $c \in \fmus$ is associated with an FMU, as defined in \cref{def:fmu}, and $\coupling(u)=y$ means that the output $y$ is connected to input $u$.
  Let $\allinputs = \bigcup_{c \in \fmus} \inputs{c}$ and $\alloutputs = \bigcup_{c \in \fmus} \outputs{c}$, then $\coupling : \allinputs \to \alloutputs$.
\end{definition}

The following definitions correspond to the operations that are permitted in the initialization phase of a co-simulation.

\begin{definition}[Step]\label{def:cosim_step}
  Given a scenario $\tuple{\fmus, \coupling}$, a co-simulation step, or just step, is a finite ordered sequence of FMU function calls $\sequence{\functioncall_i}_{i \in \setnat} = \functioncall_0, \functioncall_1, \ldots$ with
  $\functioncall_i \in \allfunctioncalls = \bigcup_{c \in \fmus} \set{\fset{c},\fget{c}},$
  and $i$ denoting the order of the function call.
\end{definition}

\begin{definition}[Initialization]\label{def:initialization}
  Given a scenario $\tuple{\fmus, \coupling}$, we define the initialization procedure $\sequence{\initcall_i}_{i \in \setnat}$ in the same way as a step, with $\initcall_i \in \allfunctioncalls$.
\end{definition}

\begin{definition}[Feed-through]\label{def:feedthrough}
  The input $\inputvar{c} \in \inputs{c}$ feeds through to output $\outputvar{c} \in \outputs{c}$, that is, $(\inputvar{c},\outputvar{c}) \in \feedthrough{c}$, when there exists $v_1, v_2 \in \values$ and $\state{c} \in \stateset{c}$, such that
  $
  \fget{c} (\fset{c}(\state{c}, \inputvar{c}, v_1), \outputvar{c}) \neq \fget{c} (\fset{c}(\state{c}, \inputvar{c}, v_2), \outputvar{c}).
  $
\end{definition}

\begin{definition}[Output Computation]\label{def:getout}
The $\fget{c}(\dontcare, \outputvar{c})$ represents the calculation of output $\outputvar{c}$ of $c \in \fmus$. Given a co-simulation state, it checks whether all inputs that feed-through to $\outputvar{c}$ are defined.
\end{definition}

\begin{definition}[Input Computation]\label{def:setin}
The $\fset{c}(\dontcare, \inputvar{c}, \inputV)$ represents the setting of input $\inputvar{c}$  of $c \in \fmus$. Given a co-simulation state, it checks whether all outputs connected to $\inputvar{c}$ are defined.
\end{definition}

\begin{definition}[Interconnected variable]
An interconnected variable v of a co-simulation scenario $\tuple{\fmus, \coupling}$ is defined as $v \in \allinputs \cup \alloutputs$, then $\coupling : \allinputs \to \alloutputs$.
The set all all interconnected variables is denoted: $V = \allinputs \cup \alloutputs$,
\end{definition}

A graph of the dependencies of a co-simulation scenario is established from the interconnected variables by \cref{def:initialization_graph}. The graph is the foundation for the calculation of the initialization procedure and is therefore referred to as the Initialization Graph.

\begin{definition}[Initialization Graph]\label{def:initialization_graph}
  Given a co-simulation scenario $\tuple{\fmus, \coupling}$, and a set of feed-through dependencies $\bigcup_{c \in \fmus} \set{\feedthrough{c}}$, we define the Initialization Graph where each node represents a port $\outputvar{c} \in \outputs{c}$ or $\inputvar{c} \in \inputs{c}$ of some fmu $c \in \fmus$. The edges are created according to the following rules:
  \begin{compactenum}
    \item For each $c \in \fmus$ and $\inputvar{c} \in \inputs{c}$, if $\coupling(\inputvar{c}) = \outputvar{d}$, add an edge $\outputvar{d} -> \inputvar{c}$.
    \item For each $c \in \fmus$ and $(\inputvar{c},\outputvar{c}) \in \feedthrough{c}$, add an edge $\inputvar{c} -> \outputvar{c}$.
  \end{compactenum}
\end{definition}

The interconnections of FMU variables can lead to circular dependencies between them, as seen in \cref{fig:fmu_cycle} showing a co-simulation scenario containing an algebraic loop and its Initialization Graph.

\begin{figure}[H]
    \centering
    \begin{minipage}{0.55\textwidth}
        \centering
    \includegraphics[width=1\textwidth]{images/fmu_cycle.pdf}
    \end{minipage}\hfill
    \begin{minipage}{0.35\textwidth}
        \centering
    \includegraphics[width=1\textwidth]{images/SCC.pdf}
    \end{minipage}
    \caption{An FMU co-simulation scenario and its Initialization Graph}
    \label{fig:fmu_cycle}
\end{figure}






The following definitions will formalize the concept of an algebraic loop in a co-simulation scenario, and define the problem these algebraic loops are introducing. 

\begin{definition}[Algebraic loops] 
An algebraic loop is defined as a strongly connected component of the graph in \cref{def:initialization_graph}. \\
A strongly connected component: $SCC = v \in V: Path(v,v)$\\
$\LoopVariables = \{v | v \in V : Path(v,v)\}$ \\
An algebraic loop is defined as the set $\{p_1, p_2 | p_1, p_2 \in \LoopVariables: Path(p_1, p_2) \land Path(p_2, p_1)\}$\todo{Look at this - can it be specified better?}\\
Path is defined as the transitive closure of the edges of the graph denoted by \cref{def:initialization_graph}.\\
\todo{Claudio check this definition and add a formalization of the challenge introduced}
\end{definition}

\begin{definition}[Challenge of Algebraic loops]\label{def:challenge}
The algebraic dependency will lead to the following difference between performing to following :
\todo{Claudio check the other definitions and add a formalization of the challenge introduced}
\end{definition}

As described by \cref{def:challenge} the algebraic loops introduce certain challenges in the system. This means that algebraic loops needs to be handled using fixed point iterations\cite{Gomes2018}. It is a technique to repeatedly perform the steps of a sub-list of the co-simulation step. The number of repetitions the operations should be be performed depends on the characteristics of the scenario. The operations should be performed until the system converge. 
An example of applying a fixed point iteration can be seen in \cref{def:fixedpoint} where a system containing an algebraic loop has to be initialized using fixed point iteration.
\claudio{Perhaps later you can ditch \cref{fig:fixedpont} and refer to the case study section, to shorten the text.}

\begin{figure}[H]
    \centering
    \includegraphics[width=0.8\textwidth]{images/fixedpoint.pdf}
    \caption{Fixed point iteration in a co-simulation scenario}
    \label{fig:fixedpont}
\end{figure}

The initialization of the system requires the application of fixed point iteration, as defined by \cref{def:fixedpoint}.
\begin{definition}[Fixed point iteration of an Initialization procedure]\label{def:fixedpoint}
Given an Initialization procedure $\sequence{\initcall_i}_{i \in \setnat}$ of a co-simulation $\tuple{\fmus, \coupling}$ where $\LoopVariables \neq \emptyset$.\\
Fixed point iteration should be used by 
$\forall v \in \LoopVariables $

\end{definition}

However a fixed point iteration technique is not guaranteed to convergence if the system is unstable. This means that an upper bound of the number of repetitions needs to be established to ensure termination. In case of a non-converging algebraic loop the simulation should be stopped since the result of the co-simulation scenario would not be trustworthy. The criteria of a valid co-simulation scenario is specified in \cref{def:convergence}.

\begin{definition}[Convergence of Fixed point iteration]\label{def:convergence}
A fixed point iteration converges if a finite number of iterations will make the difference of the output value of the same operation between two following iterations within a certain threshold.\\
$Convergence = \exists n \in \setnat: \forall p \in \LoopVariables \implies |f(p, n + 1) - f(p, n)| \leq T$\\
$f$ is the operation performed on a loop-variable p of a given iteration n.
\end{definition}

\subsection{INTO-CPS Maestro 2}\todo{We can have some more in this section - potentially half a page - Ask Casper if he can do it?}
INTO-CPS Maestro 2\footnote{currently in alpha \url{https://github.com/INTO-CPS-Association/maestro/tree/2.0.0-alpha}}\cite{thule_maestro2_2019} is an FMI-based co-simulation framework set to supersede Maestro\cite{Maestro}. The philosophy of the framework is to apply plugins to generate co-simulation specifications expressed in the domain specific language called Maestro Base Language (MaBL). Such specifications are then interpreted and executed, resolving in the execution of a co-simulation.

\section{Calculation of an Initialization Order}\label{sc:initilization}
The FMI specification defines certain information about the initialization order described through different states of a co-simulation. The initialization phase covers the two states (in chronological order) defined in the specification:
\begin{itemize}
    \item \textit{Instantiated}
    \item \textit{Initialization Mode}
\end{itemize}
In each of the two states, different groups of FMU variables are potentially assigned a value. The groups are defined by FMI based on rules of the characteristics of the variables (\textit{causality}, \textit{initial} and \textit{variability}). These rules have been extracted as predicates and used in the implementation. 
An example of a group from the FMI specification is the \textit{INI}-group that consists of FMU variables with $initial = exact\, \lor initial = \,approx $  and $variability \neq constant$. All variables of this group are set in the \textit{Instantiated}-phase of a co-simulation. Since these variables have no connections to other FMU variables - meaning they are not represented in the graph of \cref{def:initialization_graph}, the order is insignificant. 
\claudio{The above sentence is not entirely correct: variables with initial approx may still depend on other variables. The initial approx means that their values should be used as initial guesses in algebraic loop computations. Therefore, they may still be modified after entering initialization mode. To make it correct, maybe you need to ommit the condition on initial approx.}
\claudio{Also, the reader may have no idea what does initial refer to, etc... That stuff is not on the background section. So there are two approaches to solve this: you either explain what variability and initial mean in the background, or you omit them here and rewrite the paragraph in a way that the reader does not need to understand the fmi standard. Perhaps the second option is the best one here, as this seems to be a minor detail. All you seem to want to say is: there are variables and parameters whose value does not depend on other variables. These can be set before entering the initialization mode and we will ignore them.}
However, all variables of each FMU are grouped to perform the fewest possible operations in the initialization. 
\claudio{Why is the above a "However" sentence? Is it contradicting the previous sentence?}

In the \textit{Initialization Mode} state, all variables of all FMUs should be defined.
The variables with \textit{causality = parameter} of each FMU are set first, and the order they are set is again insignificant as they are independent.
Afterwards the interconnected variables should be defined, but as stated by the \cref{def:feedthrough,def:getout,def:setin} the operations \textit{get} and \textit{set} \textbf{require} a specific initialization order, and algebraic loops places even more constraints on the initialization order. All these constraints must be satisfied by the initialization procedure of the interconnected variables.
\claudio{I feel like the previous two sentences are a bit contradictory, and are probably constructed like so because you did not deal with algebraic loops upfront. When algebraic loops are not there, there must be an order. But when algebraic loops are there, it's not that there's more constraints... it's just that one has to isolate the strong components from the rest. The result is then an order between strongly connected components, but no order within each component}

\subsection{Method to calculate the initialization order}
This section describes the approach taken to calculate the initialization order of the interconnected FMU variables. The approach is based on the strategy proposed in Gomes et al. \cite{Gomes2019b, BromanCompositionCo-Simulation}, but the approach in this work is extended with the ability to handle the initialization of algebraic loops. 

\claudio{I like the way in which you've structured this subsection... but perhaps you don't need to make this structure explciit in the latex.. just leave the subsubsection comments commented out, but still follow the structure.}
\subsubsection{Structure of the graph}
The initialization algorithm starts by building a directed graph of the dependency between the interconnected variables of the FMUs. The graph is constructed based on the interconnected variables and internal connections (feed-through), as in \cref{def:initialization_graph}. 
%Each interconnected variable in the system represents a node, and the edges are based on these connections. The edges of this graph represent precedence constraints based on the algebraic dependencies of the interconnected variables. Please see \cref{def:initialization_graph} for a formal definition of the graph.

%As described earlier, not all co-simulation scenarios are suitable, and these invalid scenarios need to be identified in this phase to avoid wasting valuable development. This is accomplished by monitoring convergence of all algebraic loops. A valid co-simulation scenario must convergence by \cref{def:convergence}.
%\claudio{Wait a second... you can only monitor convergence once you have an initialization order, and you start solving the loops. Why is paragraph here if the calculation of an initialization order is still to be explained? Maybe move this explanation to after that part.}

\subsubsection{Calculation of an initialization order}
The topological ordering or the strongly connected components of the graph defined in \cref{def:initialization_graph} is the initialization order of the interconnected FMU variables. 
The non-trivial strongly connected components are algebraic loops of the system. The trivial ones are standard interconnected FMU variables, whose port operation should only be performed only once during the initialization procedure.
The calculation of an initialization order is performed in linear time based on the number of both external and internal connections using Tarjan's algorithm \cite{tarjan_1972}. 
%This algorithm is selected due to its properties. It solves two of the central issues of the initialization-phase of the co-simulation.
%\begin{itemize}
%    \item Identifies algebraic loops between interconnected variables (strongly connected components)
%    \item Performs a topological sorting of the Strongly Connected Components
%\end{itemize}
The algorithm is well-established, and there exist formal proofs of its correctness and properties\cite{stefanMerz}. 
% The algorithm is among the most efficient graph algorithms for accomplishing the defined goals.
% Tarjan's algorithm is performing a topological sorting of the strongly connected components of a graph. Moreover, it can handle both graphs with and without algebraic loops.

\subsubsection{Handling of algebraic loops}
As described earlier sections, it is essential to handle the algebraic loops by a particular initialization strategy since they otherwise would invalidate the result of the co-simulation. The strategy for managing algebraic loops is to identify and initialize them using a fixed point iteration until convergence. Since convergence is not guaranteed, this property is monitored using \cref{def:convergence}. If convergence is not established within a finite number of iteration, the co-simulation is rejected to avoid running an invalid simulation.

\subsection{Optimization of a Initialization Procedure}
An initialization procedure can be optimized since the FMI specification allows multiple \textit{set} or \textit{get} operations of the same FMU to be performed in bulk by grouping them together to a single operation over multiple variables with similar characteristic. This criteria of optimization is formalized in \cref{def:optimization}
\begin{definition}[Optimization of a Initialization procedure]\label{def:optimization}
  Given an initialization procedure $\sequence{\initcall_i}_{i \in \setnat}$ with a finite ordered sequence of FMU function calls $\functioncall_i \in \allfunctioncalls = \bigcup_{c \in \fmus} \set{\fset{c},\fget{c}},$ and $i$ denoting the order of the function call. It can be optimized if $\exists \functioncall_i, \functioncall_{i +1} \in \allfunctioncalls : \exists c \in \fmus :(\functioncall_i \in {\fset{c}} \land \functioncall_{i+1} \in {\fset{c}}) \lor (\functioncall_i \in {\fget{c}} \land \functioncall_{i+1} \in {\fget{c}})$
\end{definition}
Since an Initialization procedure is defined in the same way as other co-simulation steps (see \cref{def:initialization}), the optimization criteria described in \cref{def:optimization} is valid for an arbitrary co-simulation step. \\
The correctness of this optimization is established by the proof of using the Initialization Graph's topological ordering as the initialization order by Gomes et al. \cite{Gomes2019}. This proof is valid for this approach since the optimization does not change the structure of the Initialization Graph. \\
A limitation of this optimization strategy is that it is not guaranteed to find all potentially valid optimizations of a co-simulation scenario. Considering it works only on a specific co-simulation step (a topological order of a graph), which is not necessarily unique for a given co-simulation scenario. A more advanced optimization strategy needs to be developed to perform all viable optimizations of a co-simulation step. Another solution is to apply this optimization strategy on the set of all valid co-simulation steps - yielding a potential very inefficient initialization algorithm.

\subsection{The entire Initialization Strategy}
The pseudo-code in \cref{alg:initialization} formulates the entire initialization strategy of the interconnected variables of a co-simulation scenario.
\begin{figure}[H]
  \centering
    \begin{algorithm}[H]
    \caption{Initialization strategy for Interconnected variables}
    \label{alg:initialization}
      \begin{algorithmic}[1]
        \State $InitializationGraph \gets createGraph(connections)$
        \State $SCCS \gets Tarjan(InitializationGraph)$
        \State $OptimizeInitializationOrder(SCCS)$
        \ForEach {$SCC \in SCCS$}
            \If {$isAlgebraicLoop(SCC)$}
                \State $applyFixedPointIteration(SCC)$;
            \Else
                \State $initializeVariable(SCC)$;
            \EndIf
        \EndFor
        \end{algorithmic}
    \end{algorithm}
\end{figure}

\section{Case study}
\label{sec:case_study}
In this section, we give a simple example of a co-simulation whose correct initialization demands the solution to an algebraic loop.

We consider a co-simulation of a quarter car model \cite[Section 6.4]{Schramm2014}, illustrated in \cref{fig:quarter_car}.
We omit the equations that each FMU is solving but note that gravity acts on both wheel and chassis masses and that the origin of each mass is when the springs are not displaced.
The equations and simulation model for this example are available online\footnote{\url{https://github.com/SimplisticCode/QuarterCarCaseStudy}}. 

\begin{figure}[htb]
    \centering
    \includegraphics[width=0.8\textwidth]{images/quarter_car.pdf}
    \caption{Quarter car model co-simulation. Adapted from \cite[Section 6.4]{Schramm2014}.}
    \label{fig:quarter_car}
\end{figure}

The FMUs need initial conditions specified by equations that restrict the possible initial values for the position and velocity of the wheel and chassis masses.
\Cref{fig:init_state_0_sim} illustrates what happens when we simply set those positions and velocities to zero.
Note that, because of gravity, the car chassis bounces on the suspension wheel, with a maximum compression of about 17cm. This is most likely an invalid scenario, as the suspension of the car might not be rated to be displaced that much. In any case, the purpose of simulation studies involving quarter car models is to understand how well a suspension system absorbs shock when the car goes over a bump, not when the car is \emph{falls on the road}, which is what the simulation results in \cref{fig:init_state_0_sim} resemble.

\begin{figure}[htb]
\centering
\begin{tikzpicture}
    \begin{axis}[
        at={(0,0)},
        width= 12cm,
        height = 8cm,
        ymin=-0.2,
        ymax=0.05,
        xmin=0,
        xmax=4.1,
        grid=both,
        grid style={line width=.1pt, draw=gray!10},
        major grid style={line width=.2pt,draw=gray!50},
        axis lines=middle,
        minor tick num =5,
        axis line style={latex-latex},
        ticklabel style={font=\tiny,fill=white},
        minor tick style={draw=none},
        minor grid style={thin,color=black!10},
        ylabel= Car position,
        xlabel= Time (s),
        tick align=outside,
        x label style={at={(axis description cs:0.5,-0.1)},anchor=north},
        y label style={at={(axis description cs:-0.1,.5)},rotate=90,anchor=south},
        xlabel style={color=blue!50!cyan},
        ylabel style={align=center,rotate=-90,color=blue!50!cyan},
        x tick label style={
            /pgf/number format/assume math mode, font=\sf\scriptsize}
        ]
    \addplot [line width=2.5pt, red] table [x=a, y=b, col sep=comma] {incorrect.csv};
\end{axis}
\end{tikzpicture}
    \caption{Simulation results when position and velocity of the chassis mass is zero.}
    \label{fig:init_state_0_sim}
\end{figure}

The correct way to initialize this co-simulation scenario is to force the master algorithm to calculate the valid initial velocities and position from equations that force the accelerations and velocities on the masses to be zero.
This will force the co-simulation to initialize to a steady state.

To make the above explanation concrete, we now show the equations that are active at the initial time for each FMU for a correct initialization, and we show that there is an algebraic loop.

For the road FMU, the initial equation is simply the initial height of the road surface, which in this case is zero, i.e., $z_s=0$.
For the suspension FMU, the following equations are active:
\begin{align}
& a_R = 0.0 & \text{Acceleration of tire} \\
& v_R = 0.0 & \text{Velocity of tire} \\
& F_{gR} = 9.81 * m_R  & \text{Gravity on the tire} \\
& F_R = - c_R * z_R  & \text{Rubber force acting on tire} \\
& F_A = c_A * (z_A - z_R) + d_A * (v_A - v_r)  & \text{Suspension force acting on tire} \\
& F_{\mathit{total}} = F_R + F_A - F_{gR} & \text{Total forces acting on tire} \\
& a_R = (1/m_R) * F_{\mathit{total}}  & \text{Acceleration of tire}
\end{align}
Finally, for the Chassis FMU, the following equations are active at the initial time:
\begin{align}
& a_A = 0.0 & \text{Acceleration of chassis} \\
& v_A = 0.0 & \text{Velocity of chassis} \\
& F_{gA} = 9.81 * m_A  & \text{Gravity on the chassis} \\
& a_A = (1/m_A) * (- F_A - F_{gA})  & \text{Acceleration of chassis}
\end{align}

To see that there is an algebraic loop, note that the output $z_A$ of the chassis FMU is not restricted directly, but instead has to be computed from the acceleration equations $a_A = 0 = (1/m_A) * (- F_A - F_{gA})$.
The later contains the output $F_A$ of the Suspension FMU.
This output, in turn, depends on $z_A$, thus yielding an algebraic loop.
\Cref{fig:init_state_correct_sim} shows the simulation results when the algebraic loop is properly solved during initialization.

\begin{figure}[htb]
\centering
\begin{tikzpicture}
    \begin{axis}[
        %at={(0,0)},
        width= 12cm,
        height = 8cm,
        ymin=-0.18,
        ymax=-0.05,
        xmin=0,
        xmax=4.1,
        grid=both,
        grid style={line width=.1pt, draw=gray!10},
        major grid style={line width=.2pt,draw=gray!50},
        axis lines=middle,
        minor tick num =5,
        axis line style={latex-latex},
        ticklabel style={font=\tiny,fill=white},
        minor tick style={draw=none},
        minor grid style={thin,color=black!10},
        x label style={at={(axis description cs:0.5,-0.1)},anchor=north},
        y label style={at={(axis description cs:-0.1,.5)},rotate=90,anchor=south},
        ylabel= Car position,
        xlabel= Time (s),
        tick align=outside,
        xlabel style={color=blue!50!cyan},
        ylabel style={align=center,rotate=-90,color=blue!50!cyan},
        x tick label style={
            /pgf/number format/assume math mode, font=\sf\scriptsize}
            ]
    \addplot [line width=2.5pt, red] table [x=a, y=b, col sep=comma] {correct.csv};
\end{axis}
\end{tikzpicture}
    \caption{Simulation results starting from a correct initial state (a steady state).}
    \label{fig:init_state_correct_sim}
\end{figure}

\section{Realization of a Maestro 2 Plugin}\label{sc:implementation}
The presented approach has been realized as a Maestro 2 expansion plugin that generates the \textit{Initialization}-phase of a co-simulation specification expressed in MaBL. The plugin calculates the MaBL-specification based on the FMUs of a co-simulation scenario and a specific plugin-configuration to let the user supply the initial values of FMU parameters and fine-tune the initialization of the system. The plugin can calculate a correct initialization specification if the co-simulation scenario adheres to the behavior dictated by the definition given in \cref{def:convergence} meaning all algebraic loops in the scenario convergences within a finite number of iterations.

The plugin optimizes the initialization order by grouping operations that can be executed in \textit{parallel} to take advantage of FMI's ability to \textit{set} or \textit{get} multiple variables of a single FMU in bulk. The criteria for this optimization is defined in \cref{def:optimization}. 
The developed plugin has been tested on numerous co-simulation scenarios from the INTO-CPS universe\cite{Maestro} and compared with the existing \textit{Initializer} of Maestro. The plugin has been tested as a part of the complete Maestro 2 pipeline. 

\subsection{Realization of the Topological Sorting}
The topological sorting algorithm (Tarjan's Algorithm) is implemented in Scala\cite{Scala}, an object-oriented programming language incorporating many features from the functional programming paradigm. The motivation for choosing Scala\cite{Scala} is its relation to JVM and the connection to Slang and the Sireum framework\cite{inbook}. Slang (Sireum Language) is a programming language based on Scala, developed at Kansas State University (KSU), to develop and reason about critical software systems. Sireum is a framework for performing programming language analysis, reasoning, and verification of CPS also developed at KSU. Logika is one of the tools in the Sireum framework used for performing automated formal verification of a piece of Slang code using the theorem prover Z3\cite{Z3prover2020Sep}.
The connection of the implementation to Slang and Logika will be investigated in future work. The plan is to use the Logika framework to formally verifying the plugin. This will also be used to explore how Slang's contract-based nature can be used to obtain more reliable results of co-simulations. Tarjan's algorithm returns a topological order of strongly connected components. The returned order is the initialization order, where the non-trivial strongly connected components denote an algebraic loop requiring a particular initialization strategy. 

\subsection{Verification of the Initialization Order}
The plugin is verified using several methods. The plugin approach is established using traditional proof methods, and the plugin has been practically verified against an established co-simulation step verifier. Gomes et al. have verified the approach in \cite{gomes_lucio_vangheluwe_2019}; they proved the correctness of using the topological order of a dependency graph of the interconnected FMU-variables as the order of the operations in a co-simulation step (both the initialization procedure and an arbitrary step). Gomes et al. \cite{gomes_lucio_vangheluwe_2019} used a graph of FMU-operations (\textit{Set, Get, doStep}) in their proof instead of interconnected FMU-variables, which is the approach of this paper. The simplification of using the interconnected FMU-variables is valid and preserves the properties proved by Gomes et al. since this approach only considers the initialization phase of a co-simulation. This makes it possible to omit all the \textit{doStep} nodes from Gomes et al.'s graph, eventually ending up with a graph similar to the initialization graph described in \cref{def:initialization_graph}. 
This approach is a subgraph of the graph by Gomes et al.\cite{gomes_lucio_vangheluwe_2019}, which allows their proof to be modified to the approach presented in this paper.

\subsubsection{Practical Verification against an Established Verifier} 
Gomes et al.'s \cite{gomes_lucio_vangheluwe_2019} main contribution is a Prolog implementation of the principles for a valid FMI based co-simulation step \footnote{\url{http://msdl.cs.mcgill.ca/people/claudio/projs/PrologCosimGeneration.zip}.}. Gomes et al. use the Prolog implementation in their research to verify their approach for generating different co-simulation algorithms. The Prolog realization encapsulates all the rules of a valid co-simulation step (both master-algorithm and an initialization algorithm). 
The Initializer includes an integration to the Prolog Verifier. The integration is a Java program based on JIProlog \cite{JIProlog} - a library that allows calling Prolog predicates directly from Java. The integration is used to check the initialization order against the rules in the Prolog database. The integration performs all the necessary transformations of the dependency graph (see definition \ref{def:initialization_graph}) used in the Maestro plugin to a graph of FMU operations used in the Prolog database. The transformation is based on the definitions \ref{def:getout} and \ref{def:setin}. The integration has been realized to systematically verify the calculated initialization order's correctness against an established and recognized co-simulation Algorithm Verifier. 
The Prolog implementation does support co-simulation scenarios containing algebraic loop, so these scenarios are not tested against the Prolog database.


\section{Related Work}
Prior work \cite{Gomes2019, BromanCompositionCo-Simulation} has been looking into the generation of co-simulation algorithms for both master and initialization algorithms for FMI-based scenarios based on a dependency graph of the operations of a co-simulation step. Gomes and Broman do both present an approach where they use the topological sorting of a dependency graph to establish a correct order of operations in a co-simulation step.
Gomes et al. \cite{Gomes2019} propose a strategy for the generation of co-simulation algorithms and define the criteria for a correct co-simulation step. Their work has many similarities with ours. However, their work does not look into the handling of algebraic loops and is not implemented as a real co-simulation framework. Furthermore, the approach taken in our work is only concerned with the initialization procedure of a co-simulation.

The work by Broman et al. \cite{BromanCompositionCo-Simulation} proposes using the topological sorting of a dependency graph of the interconnected variables to detect algebraic loops and discover the partial order of port-operations. Nevertheless, they explicitly specify the requirement for cycle freedom in the dependency graph as a precondition for the generation of a deterministic co-simulation algorithm. It means they refuse all co-simulation scenarios containing algebraic loops. This is a significant difference to our approach that applies a fixed point iteration strategy to handle these scenarios. Also, the approach in this paper is more specialized in the sense that it is only considering the initialization of a co-simulation. 

Amalio et al. \cite{Amalio2016CheckingCo-simulation} has been investigating how to avoid algebraic loops in FMU based co-simulation scenarios by statically checking the architectural design of a CPS. The publication's purpose is like ours to avoid invalid co-simulation scenarios. Nevertheless, they achieve this by excluding scenarios containing algebraic loops. Their method is like ours also implemented in the real tool INTO-SysML, and it is based on using different kinds of formal methods (Theorem Proving and Model-checking) and will be an inspiration for the future work of formally verifying the plugin. 


\section{Concluding remarks}\label{sc:summary}
This work uses a topological ordering of a dependency graph of the interconnected FMUs variable and internal FMU variable connections along with predicates from the FMI specification to calculate a correct initialization order for a co-simulation scenario. The calculated initialization order is optimized to group variables with similar characteristics to perform the fewest possible operations in the initialization procedure.
This approach supports the initialization of a co-simulation scenario containing algebraic loops by utilization of fixed point iteration. The approach does also discard all co-simulation scenarios which are not adhering to the law of convergence \ref{def:convergence}.
This approach can be combined with an arbitrary master algorithm to run the co-simulation. \\
The approach was realized as a plugin for the open-source INTO-CPS Maestro 2 tool and verified against the existing \textit{Initializer} and the calculated initialization order was verified against an established co-simulation Algorithm Generator and Verifier implemented in Prolog\cite{gomes_lucio_vangheluwe_2019}\\
Future work includes formal verification of the plugin using the Logika framework\cite{inbook}.
Future work will also look into the generation of a verification strategy for the whole Maestro 2 framework to explore how different forms of verification jointly can extend the trust of the correctness of the result of a co-simulation. 

\paragraph*{\textbf{Acknowlegements.}}We would like to thank Stefan Hallerstede, Christian Møldrup Legaard and Peter Gorm Larsen for providing valuable input to this paper and the developed plugin.
%
% ---- Bibliography ---

\bibliographystyle{splncs04}
\bibliography{bib}


\end{document}
