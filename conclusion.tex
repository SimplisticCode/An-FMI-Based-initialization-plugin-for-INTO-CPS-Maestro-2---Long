\section{Concluding remarks}\label{sc:summary}
This work used a topological ordering based on the interconnected FMUs variable and internal FMU variable connections along with predicates from the FMI specification to calculate a correct initialization order for a co-simulation scenario. The calculated initialization order is optimized to group similar variables to perform fewest possible operations.
This approach ensures absence of cycles between interconnected FMU variables by defining the criteria for valid co-simulation scenarios.
This approach can be combined with an arbitrary master algorithm to run the co-simulation. \\
The approach was realized as a plugin for the open-source INTO-CPS Maestro 2 tool and verified against the existing \textit{Initializer} and the calculated initialization order was verified against an established co-simulation algorithm generator and verifier implemented in Prolog using an integration.\\
Future work includes formally verifying the plugin, and its implementation of the algorithm used to calculate the topological ordering using the Logika framework.
Future work will also look into generation of a verification strategy for the whole Maestro 2 framework to explore how different forms of verification jointly can extend the trust of the correctness of the result of a co-simulation. 

\paragraph*{\textbf{Acknowlegements.}}We would like to thank Stefan Hallerstede, Peter Gorm Larsen and Claudio Gomes for providing valuable input to this paper and the developed plugin.