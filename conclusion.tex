\section{Concluding remarks}\label{sc:summary}
This work uses a topological ordering of a dependency graph of the interconnected FMUs variable and internal FMU variable connections along with predicates from the FMI specification to calculate a correct initialization order for a co-simulation scenario. The plugin optimizes the calculated initialization order by grouping variables with similar characteristics to perform the fewest possible operations in the initialization procedure.
This approach supports the initialization of a co-simulation scenario containing algebraic loops by utilization of fixed point iteration. The approach discards all co-simulation scenarios, not adhering to the law of convergence defined in \cref{def:convergence}.
It is possible to combine the approach with an arbitrary master algorithm meaning the approach is suitable to combine with well-established master algorithms like Gauss-Seidel and Jacobian \cite{Palensky2017}. 
The approach is realized as a plugin for the open-source INTO-CPS Maestro 2 tool and verified against the existing \textit{Initializer} and the calculated initialization order was verified against an established co-simulation Algorithm Generator and Verifier implemented in Prolog\cite{gomes_lucio_vangheluwe_2019}\\
Future work includes formal verification of the plugin using the Logika framework\cite{inbook}.
We will also look into the generation of a verification strategy for the whole Maestro 2 framework to examine how different forms of verification jointly can extend the trust of the correctness of the result of a co-simulation. 

\paragraph*{\textbf{Acknowlegements.}}We would like to thank Stefan Hallerstede, Christian Møldrup Legaard, and Peter Gorm Larsen for providing valuable input to this paper and the developed plugin.