\begin{abstract}
The development of cyber-physical systems (CPS) is challenging due to the heterogeneity of the different subsystems. Co-simulation can be used to assist in the development process, where models of constituents of a CPS are coupled to simulate the behavior of the full system jointly. However, an accurate result of the co-simulation is dependent on the correct initialization of all the simulation units. In this work, we consider co-simulation where the simulation units are described as Functional Mock-up Units (FMU). The Functional Mock-up Interface specification specifies constraints to the initialization of variables in the scope of a single FMU. However, it does not consider the initialization of interconnected variables between instances of FMUs. Such interconnected variables place particular constraints on the initialization order of the FMUs.\\
The approach taken to calculate a correct initialization order is based on predicates from the FMI specification and the topological ordering of both internal connections and interconnected variables. The approach supports the initialization of algebraic loops using fixed point iteration.%  dependencies between FMU variables are dismissed. %This approach has been compared to other already existing approaches for FMI initialization. 
The approach has been realized as a plugin for the open-source INTO-CPS Maestro 2 Co-simulation framework. It has been tested for various scenarios and compared to an existing \textit{Initializer} that has been validated through academic and industrial application.% The approach has also been directly tested against an established co-simulation algorithm generator implemented in Prolog.

\keywords{Co-simulation \and Initialization \and Algebraic loop \and Topological ordering \and FMI}
\end{abstract}

%The correct initialization of a co-simulation depends not only on the value being assigned to each port but also the order in which the ports are initialized.



