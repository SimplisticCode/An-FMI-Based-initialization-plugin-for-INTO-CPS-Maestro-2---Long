\section{Related Work}
Prior work \cite{Gomes2019, BromanCompositionCo-Simulation} is looking into the generation of co-simulation algorithms (both master and initialization algorithms) for FMI-based scenarios. Their generation technique is like ours, based on a dependency graph of the operations of a co-simulation step. Both Gomes and Broman present an approach for using the topological order of a dependency graph to establish a correct order of operations in a co-simulation step of a given co-simulation scenario.
The work by Gomes et al. \cite{Gomes2019} does also define the criteria for a correct co-simulation step. Their work has many similarities with ours. However, their work is mostly concerned with the theoretical aspect of co-simulation algorithm generation and verification, while our work has a more practical nature. Gomes et al. do also not consider the handling of algebraic loops, which is a key feature of our approach. Furthermore, the approach taken in our work is only concerned with the initialization procedure of a co-simulation.

Broman et al. \cite{BromanCompositionCo-Simulation} also suggest to use the topological sorting of a dependency graph of the interconnected variables to detect algebraic loops and discover the partial order of port-operations. Nevertheless, they explicitly specify the requirement for cycle freedom in the dependency graph as a precondition for generating a valid co-simulation. It means they refuse all co-simulation scenarios containing algebraic loops. It is a significant difference to our approach that applies a fixed point iteration strategy to handle these scenarios. Also, the approach in this paper is more specialized because it only considers the initialization of a co-simulation, which means it deals with non-interconnected variables.

Amalio et al. \cite{Amalio2016CheckingCo-simulation} investigate how to avoid algebraic loops in FMU based co-simulation scenarios by statically checking the architectural design of a CPS. The publication's purpose is like ours, to avoid invalid co-simulation scenarios. Nevertheless, they achieve this by excluding co-simulation scenarios containing algebraic loops. Their method is realized in a co-simulation tool, INTO-SysML\cite{Miyazawa2016INtegratedModelling}. Formal methods form the basis of their work (Theorem Proving and Model-checking). It will be an inspiration for the future work of formally verifying the plugin and other parts of Maestro 2. 

The work by Gomez et al.\cite{EvoraGomez2019a} is similar to ours. They use Tarjan's SCC algorithm to generate a sorted DAG of strongly connected components to solve the initialization problem.
Even though their work is very similar to ours, we extend their approach with the verification against the simulation semantics resulting in a formally more sound approach. However, further work will look into further improvements and formal verification of the current approach.