\section{Related Work}
Prior work \cite{Gomes2019, BromanCompositionCo-Simulation} is looking into the generation of co-simulation algorithms (both master and initialization algorithms) for FMI-based scenarios. Their generation technique is like ours based on a dependency graph of the operations of a co-simulation step. Both Gomes and Broman present an approach for using the topological order of a dependency graph to establish a correct order of operations in a co-simulation step of a given co-simulation scenario.
The work by Gomes et al. \cite{Gomes2019} does also define the criteria for a correct co-simulation step. Their work has many similarities with ours. However, their work is mostly concerned with the theoretical aspect of co-simulation algorithm generation and verification, while our work has a more practical nature. Gomes et al. do also not consider the handling of algebraic loops, which is a key feature of our approach. Furthermore, the approach taken in our work is only concerned with the initialization procedure of a co-simulation.

Broman et al. \cite{BromanCompositionCo-Simulation} also suggest to use the topological sorting of a dependency graph of the interconnected variables to detect algebraic loops and discover the partial order of port-operations. Nevertheless, they explicitly specify the requirement for cycle freedom in the dependency graph as a precondition for the generation of a valid co-simulation. It means they refuse all co-simulation scenarios containing algebraic loops. It is a significant difference to our approach that applies a fixed point iteration strategy to handle these scenarios. Also, the approach in this paper is more specialized because it only considers the initialization of a co-simulation, which means it deals with non-interconnected variables.

Amalio et al. \cite{Amalio2016CheckingCo-simulation} has been investigating 
\claudio{Also use present tense here, since it's the paper that we're talking about, and the paper is eternal.}
how to avoid algebraic loops in FMU based co-simulation scenarios by statically checking the architectural design of a CPS. The publication's purpose is like ours to avoid invalid co-simulation scenarios. Nevertheless, they achieve this by excluding scenarios containing algebraic loops. Their method is like ours realized in a real co-simulation tool, INTO-SysML\cite{Miyazawa2016INtegratedModelling}. Formal methods form the basis of their work, where they have used different kinds of formal methods (Theorem Proving and Model-checking). It will be an inspiration for the future work of formally verifying the plugin and other parts of Maestro 2. 

\claudio{Also related: \cite{EvoraGomez2019a}. Copy paste from paper content  below (please have a look and rephrase).
A suggestion to distinguish your work from theirs: the resulting algorithm is verified against the simulation semantics (hence the more formal approach.). But this is a weak argument. 
In practice, we will probably need to do more work to really move ahead of the state of the art.
}
{\scriptsize
\begin{verbatim}
The key idea is that a topological sorting of the directed
acyclic graph (DAG) naturally gives the order in which the
variables must be initialised. Therefore, this led to study
how to convert a generic directed graph into a DAG. The
solution found is to build the graph of strongly connected
components (SCC) corresponding to cyclic dependencies.
The resulting graph in which each SCC has been con-
tracted into a single vertex is a DAG. We use Tarjan’s SCC
algorithm (Tarjan, 1972) (used in many Modelica tools) to
identify each SCC in the dependency graph (runs in lin-
ear time). Following the order obtained with a topological
sorting on the contracted SCC graph:
1. for nodes which were not contracted, simply propa-
gate their values
2. for nodes which were contracted (they correspond
to cyclic dependencies), we solve the initialisation
problem using an iterative algorithm called JNRA
(Jacobian based Newton-Raphson Algorithm) in-
spired by traditional Newton-Raphson algorithms of-
ten used for electric load flow computation.
\end{verbatim}
}