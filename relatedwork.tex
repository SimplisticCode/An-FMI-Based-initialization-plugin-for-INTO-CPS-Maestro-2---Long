\section{Related Work}
Prior work \cite{Gomes2019, BromanCompositionCo-Simulation} has been looking into the generation of co-simulation algorithms for both master and initialization algorithms for FMI-based scenarios based on a dependency graph of the operations of a co-simulation step. Gomes and Broman do both present an approach where they use the topological sorting of a dependency graph to establish a correct order of operations in a co-simulation step.
Gomes et al. \cite{Gomes2019} propose a strategy for the generation of co-simulation algorithms and define the criteria for a correct co-simulation step. Their work has many similarities with ours. However, their work does not look into the handling of algebraic loops and is not implemented as a real co-simulation framework. Furthermore, the approach taken in our work is only concerned with the initialization procedure of a co-simulation.

The work by Broman et al. \cite{BromanCompositionCo-Simulation} proposes using the topological sorting of a dependency graph of the interconnected variables to detect algebraic loops and discover the partial order of port-operations. Nevertheless, they explicitly specify the requirement for cycle freedom in the dependency graph as a precondition for the generation of a deterministic co-simulation algorithm. It means they refuse all co-simulation scenarios containing algebraic loops. This is a significant difference to our approach that applies a fixed point iteration strategy to handle these scenarios. Also, the approach in this paper is more specialized in the sense that it is only considering the initialization of a co-simulation. 

Amalio et al. \cite{Amalio2016CheckingCo-simulation} has been investigating how to avoid algebraic loops in FMU based co-simulation scenarios by statically checking the architectural design of a CPS. The publication's purpose is like ours to avoid invalid co-simulation scenarios. Nevertheless, they achieve this by excluding scenarios containing algebraic loops. Their method is like ours also implemented in the real tool INTO-SysML, and it is based on using different kinds of formal methods (Theorem Proving and Model-checking) and will be an inspiration for the future work of formally verifying the plugin. 
