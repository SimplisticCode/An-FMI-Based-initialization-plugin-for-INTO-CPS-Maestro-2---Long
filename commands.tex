% Comments

\newcommand{\claudio}[1]{%
    {\scriptsize
        \textbf{\textcolor{red}{Claudio: #1}}
    }%
}%

% Paper specific notation

\setlength{\intextsep}{10pt}

\newcommand{\load}{\ensuremath{\mathit{load}}}
\newcommand{\env}{\ensuremath{\mathit{env}}}
\newcommand{\psu}{\ensuremath{\mathit{psu}}}
\newcommand{\plant}{\ensuremath{\mathit{plant}}}
\newcommand{\ctrl}{\ensuremath{\mathit{ctrl}}}
\newcommand{\fref}{\ensuremath{\mathit{ref}}}
\newcommand{\xaft}{\ensuremath{\mathit{xaft}}}

\newcommand{\inputV}{v}
\newcommand{\consistent}{\ensuremath{\mathit{Consistent}}}
\newcommand{\remaining}{\ensuremath{\mathit{Remaining}}}
\newcommand{\dontcare}{\_}
\newcommand{\defined}{\ensuremath{\mathit{defined}}}
\newcommand{\undefined}{\ensuremath{\mathit{undefined}}}
\newcommand{\properties}{P}
\newcommand{\satisfies}{\vDash}
\newcommand{\simulator}{\mathcal{A}}
\newcommand{\Induced}[2]{\llbracket #1 \rrbracket_{#2}}
\newcommand{\timebase}{\setreal_{\geq 0}}
\newcommand{\stateset}[1]{S_{#1}}
\newcommand{\runstate}[1]{S^{R}_{#1}}
\newcommand{\state}[1]{s_{#1}}
\newcommand{\inputs}[1]{U_{#1}}
\newcommand{\inputvar}[1]{u_{#1}}
\newcommand{\outputs}[1]{Y_{#1}}
\newcommand{\outputvar}[1]{y_{#1}}
\newcommand{\values}{\mathcal{V}}
\newcommand{\true}{\mathit{true}}
\newcommand{\false}{\mathit{false}}
\newcommand{\feedthrough}[1]{D_{#1}}
\newcommand{\reactivity}[1]{R_{#1}}
\newcommand{\fixedpoint}[1]{\mathtt{fixedpoint}_{#1}}
\newcommand{\fset}[1]{\mathtt{set}_{#1}}
\newcommand{\fget}[1]{\mathtt{get}_{#1}}
\newcommand{\fdoStep}[1]{\mathtt{doStep}_{#1}}
\newcommand{\timestamp}[1]{\varphi(#1)}
\newcommand{\feedsto}[2]{U_{#1}^{#2}}
\newcommand{\master}{\mathcal{A}}
\newcommand{\loops}{\mathit{SCC}}

\newcommand{\LoopVariables}{LV}
\newcommand{\alloutputs}{Y}
\newcommand{\allfeedthroughs}{D}
\newcommand{\allcontracts}{\mathcal{C}}
\newcommand{\coupling}{L}
\newcommand{\allinputs}{U}
\newcommand{\fmus}{C}
\newcommand{\sequence}[1]{\pargroup{#1}}
\newcommand{\functioncall}{f}
\newcommand{\initcall}{I}
\newcommand{\allfunctioncalls}{F}
\newcommand{\fmu}[1]{\texttt{#1}}
\newcommand{\signal}[1]{\texttt{#1}}
\newcommand{\before}[2]{\ensuremath{#1 \twoheadrightarrow #2}}
\newcommand{\ibefore}[2]{\ensuremath{#1 \rightarrow #2}}
\newcommand{\after}[1]{{#1}'}
\newcommand{\aftern}[2]{{#1}^{(#2)}}
\newcommand{\stateafter}[2]{\ensuremath{\state{#1}^{(#2)}}}


\newtheorem{procedure}{Procedure}{}
\newtheorem{assumption}{Assumption}{}
%\newtheorem{problem}{Problem}{}

\theoremstyle{definition}
%\newtheorem{definition}{Definition}{}
%\newtheorem{example}{Example}{}
\newtheorem{experiment}{Experiment}{}

%\theoremstyle{remark}
%\newtheorem{remark}{Remark}{}




% Generic stuff

\newcommand{\footurl}[1]{\footnote{\url{#1}}}

% Math
\newcommand{\brackets}[1]{\ensuremath{ \left[ #1 \right] }}
\newcommand{\tuple}[1]{\ensuremath{ \left\langle #1 \right\rangle }}
\newcommand{\set}[1]{\ensuremath{ \left\{ #1 \right\}}}
\newcommand{\system}[1]{\ensuremath{ \begin{cases} #1 \end{cases}}}
\newcommand{\rightgroup}[1]{\ensuremath{ \left. \begin{matrix} #1 \end{matrix} \right\} } }
\newcommand{\pargroup}[1]{\ensuremath{ \left( #1 \right)}}
\newcommand{\inv}[1]{\ensuremath{\pargroup{ #1 }^{-1}}}
\newcommand{\dert}[1]{\ensuremath{ \dot{#1} }}
\newcommand{\ddert}[1]{\ensuremath{ \ddot{#1} }}
\newcommand{\partialder}[2]{\ensuremath{ \frac{\partial#1}{\partial#2} }}
\newcommand{\setreal}[0]{\ensuremath{\mathbb{R}}}
%\newcommand{\setbool}[0]{\ensuremath{\mathit{Bool}}}
\newcommand{\setnat}[0]{\ensuremath{\mathbb{N}}}
\newcommand{\norm}[1]{\left\lVert#1\right\rVert}
\newcommand{\bnorm}[1]{\big\lVert#1\big\rVert}
\newcommand{\abs}[1]{\left|#1\right\|}
\newcommand{\xs}[2]{\ensuremath{#1^{\left[#2\right]}}}
\newcommand{\infinitynorm}[1]{\left\lVert#1\right\rVert_\infty}
\newcommand{\bigO}[1]{\ensuremath{ \mathcal{O}\left( #1 \right)}}
\algnewcommand\algorithmicforeach{\textbf{for each:}}
\algdef{S}[FOR]{ForEach}[1]{\algorithmicforeach\ #1\ \algorithmicdo}
\newcommand{\vectorOne}[1]{\brackets{%
\begin{matrix}
  #1
 \end{matrix}%
}}
\newcommand{\vectorTwo}[2]{\brackets{%
\begin{matrix}
  #1 \\
  #2
 \end{matrix}%
}}
\newcommand{\vectorThree}[3]{\brackets{%
\begin{matrix}
  #1 \\
  #2 \\
  #3
 \end{matrix}%
}}
\newcommand{\vectorFour}[4]{\brackets{%
\begin{matrix}
  #1 \\
  #2 \\
  #3 \\
  #4
 \end{matrix}%
}}
\newcommand{\vectorFive}[5]{\brackets{%
\begin{matrix}
  #1 \\
  #2 \\
  #3 \\
  #4 \\
  #5
 \end{matrix}%
}}
\newcommand{\vectorSix}[6]{\brackets{%
\begin{matrix}
  #1 \\
  #2 \\
  #3 \\
  #4 \\
  #5 \\
  #6
 \end{matrix}%
}}
\newcommand{\vectorSeven}[7]{\brackets{%
\begin{matrix}
  #1 \\
  #2 \\
  #3 \\
  #4 \\
  #5 \\
  #6 \\
  #7
 \end{matrix}%
}}
\newcommand{\vectorEight}[8]{\brackets{%
\begin{matrix}
  #1 \\
  #2 \\
  #3 \\
  #4 \\
  #5 \\
  #6 \\
  #7 \\
  #8
 \end{matrix}%
}}

\newenvironment{aligneq*}%
{
\begin{equation*}
\begin{aligned}
}{
\end{aligned}
\end{equation*}
}

\newenvironment{aligneq}%
{
\begin{equation}
\begin{aligned}
}{
\end{aligned}
\end{equation}
}



%enable \cref{...} and \Cref{...} instead of \ref: Type of reference included in the link
%Nice formats for \cref
\usepackage{iflang}
\IfLanguageName{ngerman}{
  \crefname{table}{Tab.}{Tab.}
  \Crefname{table}{Tabelle}{Tabellen}
  \crefname{figure}{\figurename}{\figurename}
  \Crefname{figure}{Abbildungen}{Abbildungen}
  \crefname{equation}{Gleichung}{Gleichungen}
  \Crefname{equation}{Gleichung}{Gleichungen}
  \crefname{listing}{\lstlistingname}{\lstlistingname}
  \Crefname{listing}{Listing}{Listings}
  \crefname{section}{Abschnitt}{Abschnitte}
  \Crefname{section}{Abschnitt}{Abschnitte}
  \crefname{paragraph}{Abschnitt}{Abschnitte}
  \Crefname{paragraph}{Abschnitt}{Abschnitte}
  \crefname{subparagraph}{Abschnitt}{Abschnitte}
  \Crefname{subparagraph}{Abschnitt}{Abschnitte}
}{
  \crefname{section}{Sect.}{Sect.}
  \Crefname{section}{Section}{Sections}
  \crefname{listing}{\lstlistingname}{\lstlistingname}
  \Crefname{listing}{Listing}{Listings}
}
\crefname{section}{Sect.}{Sect.}
\Crefname{section}{Section}{Sections}
\crefname{listing}{\lstlistingname}{\lstlistingname}
\Crefname{listing}{Listing}{Listings}

