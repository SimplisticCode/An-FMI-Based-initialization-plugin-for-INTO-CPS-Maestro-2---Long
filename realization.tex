\section{Realization of a Maestro 2 Plugin}\label{sc:implementation}
This approach has been realized as a Maestro 2 plugin that generates the \textit{Initialization}-phase of a co-simulation specification expressed in MaBL. It calculates the specification based on a list of FMU-components of a co-simulation scenario, and a specific plugin-configuration to let the user supply initial values of FMU variables with $causality=parameter$. A correction initialization specification can be calculated by the plugin if the co-simulation scenario is adhering to definition \ref{def:convergence}.

The plugin optimizes the initialization order by grouping operations that can be executed in \textit{parallel} to take advantage of FMI's ability to set or get multiple variables in bulk from the same FMU. The criteria for this optimization is defined in definition \ref{def:optimization}. 
The developed plugin has been tested on numerous co-simulation examples from the INTO-CPS universe\cite{Maestro} and compared with the existing \textit{Initializer} of Maestro. The plugin has been tested in the whole Maestro 2 pipeline to practically verify that it can be used in a co-simulation where the generated initialization procedure is integrated as a part of a complete co-simulation to verify the initialization procedure's integration with a master algorithm.

\subsection{Realization of the Topological Sorting}
The topological sorting algorithm (Tarjan's Algorithm) was implemented in Scala. The motivation for choosing Scala is its relation to JVM and the connection to Slang and the Logika framework\cite{inbook}. The connection to Slang and Logika will be investigated in the future. The Logika framework will be used to formally verify the implementation plugin and be used to explore how the contract-based nature of Slang can be used to obtain more reliable results of co-simulations. The topological sorting returns a topological order of strongly connected components. This order is the order of initialization, where the elements containing more than a single element denotes an algebraic loop that requires a particular initialization strategy. 

\subsection{Verification of the approach}
The realized plugin and the developed approach have been verified in several different ways. Gomes et al. have in \cite{gomes_lucio_vangheluwe_2019} proved the correctness of using the topological order of a dependency graph of the interconnected variables to generate not only the initialization order but also the order for a co-simulation master-algorithm. The purpose of the publication by Gomes et al. is a little different from the one of this paper. Gomes et al. are interested in both generating an initialization and master-algorithm, meaning they also need to focus on the time of each FMU. They also define their graph based on FMU-operations ($Set, Get, doStep$) instead of interconnected variables, which is the approach described in this paper. This simplification is valid since this approach only cares about initialization and, therefore, does not need to care about the time of each FMU. This does that we can omit all the $doStep$ nodes and all edges connected to them from the graph by Gomes et al., this approach does, therefore, end up with a subgraph of the graph by Gomes et al. The proof from \cite{gomes_lucio_vangheluwe_2019} can therefore trivial be modified to the graph/approach presented in this paper.

Gomes et al.'s main contribution is an implementation of the FMI specification in Prolog freely available to be used in other projects. The Prolog implementation was used in their research to verify their proposed approach for the generation of co-simulation algorithms. The realization encapsulates all the rules of a valid master-algorithm and an initialization algorithm. As a part of this work, an integration from this plugin has been realized to interact with the Prolog realization. This integration has been realized in order to verify the correctness of the calculated initialization order against an established and recognized method. The integration is querying the Prolog database with a co-simulation environment and an initialization order. The integration requires a transformation of the dependency graph (see definition \ref{def:initialization_graph}) used in the plugin to a graph of FMU operations. This transformation is based on the definitions \ref{def:getout} and \ref{def:setin}. The described verification integration is not utilized if the graph contains algebraic loops since the Prolog database does not support these scenarios.

The Prolog code can generate an initialization order, but this functionality has not been used for several reasons. The main reason for not generating the initialization order using the Prolog code is based on performance considerations. The approach taken in this paper does also identifies the strongly connected components and performs a particular initialization strategy of these.